\documentclass[a4paper, 12pt]{article}

\usepackage{resources}

\addbibresource{} %TODO: Hier Quellenressource einfügen

\begin{document}
\begin{titlepage}
    \vspace*{0.5cm}
    \centering
    
    {\Huge [Hier Projektarbeitstitel einfügen]\par}
    \vspace{1.5cm}
    {\huge Projektarbeit\par}
    \vspace{1.25cm}
    {\Large des Studiengangs [Hier Studiengang einfügen]\par}
    {\Large an der [Hier Hochschule einfügen]\par}
    \vspace{1cm}
    {\Large von \par}
    {\Large [Hier Autor einfügen] \par}
    \vspace{1cm}
    {\Large Abgabedatum \quad [Hier Abgabedatum einfügen] \par}
    \vspace{3cm}
    {\large
        \begin{tabbing}
            \textbf{Bearbeitungszeitraum} \qquad \qquad \qquad \= [Hier Bearbeitungszeitraum einfügen]\\
            \textbf{Matrikelnummer, Kurs} \> [Hier Matrikelnummer und Kurs einfügen] \\
            \textbf{Ausbildungsbetrieb} \> [Hier Ausbildungsfirma einfügen] \\
            \textbf{Betrieblicher Betreuer} \> [Hier Betreuer einfügen]
        \end{tabbing}
    \par}
    \vspace{1cm}
    \vfill
\end{titlepage}

\pagebreak
%TODO: Hier Bestätigung Praxisphase einfügen
%\includepdf[pages=-]{../Bestätigung_Praxisphasen_Lukas_Schick.pdf}
\pagebreak
\section*{Sperrvermerk}
Die vorliegende Projektarbeit mit dem Titel [Hier Projektarbeitstitel einfügen] beinhaltet interne, vertrauliche Informationen und Daten der [Hier Firmennamen einfügen].
Diese Projektarbeit darf nur vom Erst- und Zweitgutachter sowie berechtigten Mitgliedern des Prüfungsausschusses eingesehen werden. 
Eine Vervielfältigung und Veröffentlichung der Projektarbeit ist auch auszugsweise nicht erlaubt. 
Die Verwendung der Arbeit in eventuellen prüfungsrechtlichen Rechtsschutzverfahren nach Maßgabe der geltenden verwaltungsprozessualen Regelungen ist abgesehen davon möglich.
Eine Ausnahme von dieser Regelung bedarf einer ausdrücklichen Genehmigung der [Hier Firmennamen einfügen].


\section*{Selbstständigkeitserklärung}
Ich versichere hiermit, dass ich meine Projektarbeit mit dem Titel: [Hier Projektarbeitstitel einfügen] selbstständig verfasst und keine anderen als die angegebenen Quellen und Hilfsmittel benutzt habe. 
Ich versichere zudem, dass die eingereichte elektronische Fassung mit der gedruckten Fassung übereinstimmt.

\vspace{1.25cm}
\begin{flushleft}
    \begin{minipage}[c]{0.49\textwidth}
        \begin{tabular}{@{}p{2in}@{}}
            \hrulefill \\
            Ort, Datum \\
        \end{tabular}
    \end{minipage}
    \begin{minipage}[c]{0.49\textwidth}
        \begin{tabular}{@{}p{2in}@{}}
            \hrulefill \\
            Unterschrift \\
        \end{tabular}
    \end{minipage}
\end{flushleft}

\pagebreak
\section*{Abkürzungsverzeichnis}
\begin{acronym}
    %\acro{Tag for linking}[abbreviation]{abbreviation written out}
\end{acronym}
\pagebreak
\tableofcontents
\pagebreak
\listoffigures
\renewcommand\listoflistingscaption{Quellcodeverzeichnis}
\listoflistings
\pagebreak

\begin{refsection}
% TODO: Kapitel durch mehrere Sub-Files dem Dokument hinzufügen 
%\subfile{file-path}

\pagebreak
% Quellenverzeichnis
\emergencystretch 2em
\printbibliography
\pagebreak
\end{refsection}

\phantomsection
% Um PDFs anzuhängen, falls nötig

\end{document}